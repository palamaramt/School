\documentclass{article}
\usepackage[margin = 0.15in,landscape]{geometry}
\usepackage{multicol}
\usepackage{array}
\usepackage{amsmath}
\usepackage{amssymb}
\usepackage{lmodern}
\usepackage{graphicx}
\usepackage{enumitem}
\setlength\parindent{0pt}
\renewcommand{\baselinestretch}{0.75}


\begin{document}
\begin{multicols*}{3}
    Marissa Palamara\par 
    ASEN 3200\par 
    Fall 2020
    \vspace{-0.5cm}
    \setlist{nolistsep}

    % ----- The Two-Body Problem ----- %
    \subsection*{Two-Body Problem}
    Newton's Law: $\Sigma \vec{F}=\frac{d(m\vec{v})}{dt}=m\vec{a}$ \par
    Universal Law of Gravitation: $\vec{F}_g=-\frac{Gm_1m_2}{r^2}\frac{\vec{r}}{|\vec{r}|}$\par 
    Apply Newton's Laws to a two-body problem with the assumptions:
    \begin{enumerate}
        \itemsep0em
        \item Only system force: Gravity $\rightarrow$ acts along the line joining the centers of the bodies.
        \item Mass of each body is constant.
        \item Treat each body as a spherically symmetrical point mass with uniform density.
    \end{enumerate}
    \subsection*{Orbits}
    \textbf{Elliptical Orbits}:\par
    \textbf{Orbital Properties:}
    \begin{itemize}
        \itemsep0em
        \item a = semimajor axis
        \item b = semiminor axis
        \item p = semiperimeter
        \item $r_a/r_p$ = radii of apoapsis/periapsis
        \item $\vec{e}$ = eccentricity
    \end{itemize}
    \includegraphics[width=\linewidth]{Figures/orbital_properties.png}

    \textbf{Useful Equations:}\par
    $a = \frac{1}{2}(r_a+r_p)$\par 
    $p=\frac{b^2}{a}=a(1-e^2)=\frac{h^2}{\mu}$\par 
    $r_a = \frac{p}{1-e}=a(1+e)$\par 
    $r_p = \frac{p}{1+e}=a(1-e)$\par
    $e = \frac{r_a-r_p}{r_a+r_p}$ = $\frac{c}{a}$ = $\frac{\sqrt{a^2-b^2}}{a}$ = $\sqrt{1+\frac{2h^2\varepsilon}{\mu^2}}$\par 
    $b = a\sqrt{1-e^2}$ \par
    Angular Momentum: $\vec{h} = \vec{r} \times \vec{v}$ = $\sqrt{\mu a (1-e^2)}$ \par
    Eccentricity Vector: $\vec{e} = \frac{\vec{v} \times \vec{h}}{\mu}-\frac{\vec{r}}{r}$\par
    Specific Energy: $\varepsilon = \frac{v^2}{2}-\frac{\mu}{r}$ = $\frac{\mu^2(e^2-1)}{2h^2}$\par
    \begin{itemize}
        \itemsep0em
        \item $\varepsilon < 0$ Motion of Body 2 is bounded wrt Body 1
        \item $\varepsilon \ge 0$ Motion of Body 2 is unbounded wrt Body 1
    \end{itemize}
    Conic Equation: $r = \frac{h^2/\mu}{1+e\cos{\theta}}$ = $\frac{p}{1+e\cos{\theta}}$\par 
    Vis-Viva Equation: $v=\sqrt{\frac{2\mu}{r}-\frac{\mu}{a}}$\par
    %\vspace{1.5cm}
    \textbf{True Anomaly:} 
    \begin{itemize}
        \itemsep0em
        \item $\theta$ or $\nu$
        \item Measured from periapsis, $\vec{e}$ to radius, $\vec{r}$
        \item $\theta = 0$ at periapsis
        \item $0^\circ > \theta > 180^\circ \rightarrow m_2$ moving away from periapsis
        \item $180^\circ < \theta < 360^\circ m_2$ moving toward periapsis
    \end{itemize}
    \textbf{Flight Path Angle:} 
    \begin{itemize}
        \itemsep0em
        \item $\tan{\gamma} = \frac{e\sin{\theta}}{1+e\cos{\theta}} = \frac{v_r}{v_{\theta}}$
        \item $v_r = \frac{\mu}{h}e\sin{\theta}$ and $v_\theta = \frac{\mu}{h}(1+e\cos{\theta})$
        \item $\gamma > 0$ when $v_r>0$ and $\theta>0\rightarrow$ $m_2$ moving away from periapsis
    \end{itemize}
     
    \textbf{Perifocal Frame:}  
    \begin{itemize}
        \itemsep0em
        \item $\hat{p}$, $\hat{q}$, $\hat{w}$
        \item $\vec{r} = r\hat{r} = r\cos{\theta}\hat{p} + r\sin{\theta}\hat{q}$
        \item $\vec{v} = \frac{\mu}{h}[-\sin{\theta}\hat{p}+(e+\cos{\theta})\hat{q}]$
    \end{itemize}
    \includegraphics[width=0.5\linewidth]{Figures/perifocal_plane.png}
    
    \textbf{Kepler's Law:} A line joining a planet and the Sun sweep out equal areas during equal intervals of time.\par 
    \textbf{Elliptical Orbits:} $\mathbb{P}=2\pi\sqrt{\frac{a^3}{\mu}}$, $\varepsilon<0$\par
    \textbf{Mean Motion:} $n=\sqrt{\frac{\mu}{a^3}}$ - mean angular rate of motion\par 
    \textbf{Circular Orbits:}
    \begin{itemize}
        \itemsep0em
        \item $r = a$, $\vec{v} \perp \vec{r}$, and $\gamma = 0$ everywhere
        \item $v_c=\sqrt{\frac{\mu}{r}}$
    \end{itemize}
    \textbf{Parabolic Orbits:}
    \begin{itemize}
        \itemsep0em
        \item $e=1$, $a=\inf$, $r_a=$ undefined
        \item Conic equation applies still
        \item $p = \frac{h^2}{\mu}$
        \item $\varepsilon=0$ everywhere
        \item $v = \sqrt{\frac{2\mu}{r}}=v_{esc}$
    \end{itemize}
    \textbf{Hyperbolic Orbits:}
    \begin{itemize}
        \itemsep0em
        \item $v>v_{esc}$, $e>1$, $\varepsilon>0$, $a<0$
        \item $r_p=|a|(e-1)$
        \item $p=|a|(e^2-1)=a(1-e^2)$
        \item $r=\frac{a(1-e^2)}{1+e\cos{\theta}}=\frac{|a|(e^2-1)}{1+e\cos{\theta}}$
        \item at $r = \infty$, $\varepsilon = \frac{-\mu}{2a}=\frac{v^2_{\infty}}{2}\rightarrow v_{\infty}=\sqrt{\frac{\mu}{|a|}}$
        \item $\theta_{\infty} = \pm\cos^{-1}\left({\frac{-1}{e}}\right)$
        \item $v^2=v^2_{esc}+v^2_{\infty}$
        \item turning angle: $\frac{\delta}{2}+90^\circ=\theta_\infty$, $\delta = 2\sin^{-1}({\frac{1}{e}})$
    \end{itemize}
    \includegraphics[width=0.5\linewidth]{Figures/hyperbolic_orbit.png}

    % ----- The Anomalies ----- %
    \subsection*{The Anomalies}
    \includegraphics[width=\linewidth]{Figures/The_Anomalies.png}
    \begin{itemize}
        \item True anomaly
        \begin{itemize}
            \item Advances quickly from periapsis
            \item Advances slowly from apoapsis
        \end{itemize}
        \item Mean anomaly, $M$
        \begin{itemize}
            \item Computed from time and mean motion
            \item $M = n(t-t_p)$
            \item Advances at constant rate in elliptical orbit
        \end{itemize}
        \item Eccentric anomaly, $E$
        \begin{itemize}
            \item Angle that helps translate from true anomaly to mean anomaly
            \item Advances at rate between True and Mean anomaly 
        \end{itemize}
    \end{itemize}

    \textbf{Kepler's Equation:} $M=n(t-t_p)=E-e\sin{E}$\par 
    Everything in rads!

    Finding E:\par 
    \begin{itemize}
        \item Choose an initial estimate of E:
        \begin{itemize}
            \item $M_e < \pi \rightarrow E = M_e + e/2 \quad M_e > \pi \rightarrow M_e - e/2$
        \end{itemize}
        \item $f(E) = E-e\sin{E}-M_e$
        \item $f'(E) = 1-e\cos{E}$
        \item Iterate and update $E_i$\par 
        $E_{i+1} = E_i - \frac{f(E_i)}{f'(E_i)}$ 
    \end{itemize}

    Switching between the Anomalies:\par
    \begin{itemize}
        \item $\cos{E} = \frac{e+\cos{\theta}}{1+e\cos{\theta}}$
        \item $\tan{\frac{E}{2}}=\sqrt{\frac{1-e}{1+e}}\tan{\frac{\theta}{2}}$
        \item $\tan{\frac{\theta}{2}}=\sqrt{\frac{1-e}{1+e}}\tan{\frac{E}{2}}$
        \item $r = a(1-e\cos{E})=\frac{p}{1+e\cos{\theta}}=\frac{a(1-e^2)}{1+e\cos{\theta}}$
        \item $\sin{\theta}=\frac{b}{r}\sin{E}$
    \end{itemize}

    To find time between A and B, with $a$,$e$,and $\theta$ of A and B known,\par 
    $t_B-t_A=(t_B-t_p)-(t_A-t_p)$
    \newpage
    \textbf{Hyperbolic Anomaly, H}\par 
    Use an equilateral hyperbola to determine H: $e = \sqrt{2}$\par 
    \begin{itemize}
        \item $M_h=\sqrt{\frac{\mu}{|a|^3}}(t-t_p)=e\sinh{H}-H$
        \item $\tanh{\frac{H}{2}}=\sqrt{\frac{e-1}{e+1}}\tan{\frac{\theta^*}{2}}$
        \item $\tanh{\frac{\theta^*}{2}}=\sqrt{\frac{e-1}{e+1}}\tan{\frac{H}{2}}$
    \end{itemize}

    \textbf{Parabolic Orbits}\par 
    Barkers Equation: 
        \begin{equation*}
            \begin{array}{l}
                \bullet \sqrt{\frac{\mu}{p^3}}(t-t_p)=\frac{\mu^2}{h^2}(t-t_p)=M_p=\frac{1}{6}\tan^3{\frac{\theta}{2}}+\frac{1}{2}\tan{\frac{\theta}{2}}\\
                \bullet \tan{\frac{\theta}{2}}=\left(3M_p+\sqrt{(3M_p)^2+1}\right)^{1/3}\\
                -\left(3M_p+\sqrt{(3M_p)^2+1}\right)^{-1/3}
            \end{array}
        \end{equation*}
    
    % ----- 3D Orbits ----- %
    \subsection*{3D Orbits}
    How many variables required to completely describe the state of a satellite?\par 
    6: 3 position and 3 velocity\par 
    OR\par 
    Can also describe a satellite's states by set of orbital elements:
    \begin{itemize}
        \item Size and shape: $a,e$
        \item Orientation of the orbit plane: $i,\Omega$
        \item Orientation of the orbit within the orbit plane: $\omega$
        \item Location of the satellite on the orbit: $\theta$ $(M,E,t-t_p)$
    \end{itemize}

    \textbf{Inclination, $i$:}\par 
    \begin{itemize}
        \item Angular tile of the orbital plane relative to $\hat{X}\hat{Y}$ and measured 
        between orbit normal, $\hat{h}$ and $\hat{Z}$
        \item Equatorial Orbit: $i=0^\circ, 180^\circ$
        \item Polar Orbit: $i=90^\circ$
        \item Prograde Orbit: $i=[0^\circ,90^\circ]$
        \item Retrograde Orbit: $i=[90^\circ,180^\circ]$
        \item $\cos{i}=\frac{\hat{Z}\cdot\vec{h}}{|\hat{Z}||\vec{h}|}\rightarrow\cos{i}=\frac{h_z}{h}$
    \end{itemize}

    \textbf{Right Ascension of the Ascending Node, $\Omega$:}
    \begin{itemize}
        \item Angle from the reference direction, $\hat{X}$, to the ascending node.
        \item Line of Nodes: $\vec{n}=\hat{Z}\times\vec{h}$
        \item $\cos{\Omega}=\frac{\hat{X}\cdot\vec{n}}{|\hat{X}||\vec{n}}=\frac{n_x}{n}$
        \item Quadrant Check:
            \begin{itemize}
                \item $\vec{n}\cdot\hat{Y}>0\rightarrow 0 < \Omega < 180^\circ$
                \item $\vec{n}\cdot\hat{Y}<0\rightarrow 180^\circ < \Omega < 360^\circ$
                \item Between $0-2\pi$
            \end{itemize}
    \end{itemize}

    \textbf{Argument of Periapsis, $\omega$ (AOP):}
    \begin{itemize}
        \item Measured between the lines of nodes $\vec{n}$ and the eccentricity vector, $\vec{e}$
        \item Locates the closest point of the orbit
        \item Measured within the plane, varies from $0-2\pi$
        \item $\cos{\omega}=\frac{\vec{n}\cdot\vec{e}}{|\vec{n}||\vec{e}|}$
        \item Quadrant Check:
            \begin{itemize}
                \item $\vec{e}\cdot\hat{z}>0\rightarrow 0<\omega<180^\circ$
                \item $\vec{e}\cdot\hat{z}<0\rightarrow 180^\circ<\omega<360^\circ$
            \end{itemize}
    \end{itemize}
    \vspace{1cm}
    \textbf{True Anomaly, $\theta$:}
    \begin{itemize}
        \item Location of the spacecraft within the orbit
        \item Varies from $0-2\pi$
        \item $\vec{r}\cdot\vec{e}=|\vec{r}||\vec{e}|\cos{\theta}$
        \item $\cos{\theta}=\frac{\vec{r}\cdot\vec{e}}{|\vec{r}||\vec{e}|}$
        \item Quadrant Check:
            \begin{itemize}
                \item $\vec{r}\cdot\vec{v}>0\rightarrow 0 < \theta < 180^\circ$
                \item $\vec{r}\cdot\vec{v}<0\rightarrow 180^\circ<\theta<360^\circ$
            \end{itemize}
    \end{itemize}

    \includegraphics[width=\linewidth]{Figures/Orbital_Elements.png}

    \textbf{Converting Position and Velocity to Orbital Elements:}\par
    \begin{itemize}
        \item Given $X = [\vec{r},\vec{v}]$
        \item Compute vectors and their magnitudes:
        \begin{equation*}
            \begin{array}{lll}
                \vec{h}=\vec{r} \times \vec{v} & \vec{n} = \hat{Z} \times \vec{h} & \vec{e}=\frac{\vec{v} \times \vec{h}}{\mu}-\frac{\vec{r}}{r}\\
                h=|\vec{h}| & n=|\vec{n}| & e=|\vec{e}|
            \end{array}
        \end{equation*}
        \item Compute energy to get $a$:
        \begin{equation*}
            \begin{array}{ll}
                \varepsilon=\frac{v^2}{2}-\frac{\mu}{r} & a=-\frac{\mu}{2\varepsilon}\\
                p=\frac{h^2}{\mu} & a = \frac{p}{1-e^2}
            \end{array}
        \end{equation*}
        \item Compute inclination \& Orientation Angles from above
            \begin{itemize}
                \item $\Omega\rightarrow \text{If} (n_y<0), \Omega=360^\circ-\Omega$
                \item $\omega\rightarrow \text{If} (e_z<0), \omega=360^\circ-\omega$
                \item $\theta\rightarrow \text{If} (\vec{r}\cdot\vec{v}<0), \theta=360^\circ-\theta$
            \end{itemize}
    \end{itemize}

    \textbf{What about...?}
    \begin{itemize}
        \item What is $\Omega$ for an elliptical equatorial orbit?
            \begin{itemize}
                \item Undefined, used true longitude of periapse: $\tilde{\omega}_{true}$
                \item $\cos{\tilde{\omega}_{true}}=\frac{\hat{X}\cdot\vec{e}}{|\hat{X}||\vec{e}|}$ where $\hat{X}=[1,0,0]$
                \item If $e_y<0\rightarrow \tilde{\omega}_{true}=360^\circ-\tilde{\omega}_{true}$
            \end{itemize}
        \item What is $\omega$ for a circular orbit? (No perigee) $\rightarrow$ undefined!
        \item Use arugment of latitude
            \begin{itemize}
                \item $\cos{u}=\frac{\vec{n}\cdot\vec{r}}{|\vec{n}||\vec{r}|}\rightarrow \text{If } (r_z<0), \text{then } u = 360^\circ -u$
            \end{itemize}
        \item Circular, equatorial orbits? Both $\omega$ and $\Omega$ undefined!
            \begin{itemize}
                \item $\cos{\lambda_{true}}=\frac{\hat{X}\cdot{\vec{r}}}{|\hat{X}||\vec{r}|}\rightarrow \text{If } (r_y<0), \text{then } \lambda_{true}=360^\circ - \lambda_{true}$
            \end{itemize}
    \end{itemize}

    % ----- Coordinate Frames ----- %
    \subsection*{Coordinate Frames}
    \textbf{Ecliptic Plane:}\par 
    



\end{multicols*}  
\end{document}